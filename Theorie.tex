\chapter{Das Praktikum}

\section{Fachpraktikum}

Für das Fachpraktikum mussten noch 14 Wochen absolviert werden. Alle Bereiche des Grundpraktikums wurden bereits im Vorpraktikum abgearbeitet, weshalb sich das Fachpraktikum nur auf das in den Praktikumsrichtlinien der RWTH Aachen genannte Fachpraktikum Teil A und Teil B bezieht.\\
%\cite{Praktikumsrichtlinien der RWTH Maschbau}
Die 8 Wochen aus Teil B konnten alle mit FP7 abgegolten werden, da in Weissach ausschließlich Fahrzeugentwicklung betrieben wird. Die übrigen sechs Wochen aus Teil A wurden mit FP4 - \textbf{Messen, Prüfen, Qualitätskontrolle} - mit einem Anteil von drei (3) Wochen und FP6 - \textbf{Montage} - mit weiteren drei Wochen ausgeführt.\\

\section{Fachbereich Porsche}

Bei Porsche wurde das Praktikum im Fachbereich \textbf{Aerodynamik und Thermomanagement} absolviert. \\
Diese Abteilung ist für die thermische Absicherung der Fahrzeuge und die aerodynamische Auslegung der Neuentwicklungen zuständig.\\
Neben Versuchen im Klimawind- oder Windkanal werden auch Erprobungsfahrten, Simulationen und andere Mess- und Testmethoden verwendet um die hohen Standards bei Porsche aufrecht zu erhalten. \\
Im engen Austausch mit den anderen Abteilungen und Bauteilverantwortlichen wird so jedes Fahrzeug und Einbauteil erprobt und entsprechend der geltenden Normen und Rechtssprechungen, sowie der Anforderungen die Porsche an sich selbst stellt, entwickelt und zur Serienreife gebracht.

