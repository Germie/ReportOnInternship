\chapter{Die Porsche AG}
\label{chap:Intro}

Porsche wurde 1931 von Ferdinand Porsche gegründet. Seit 1945 stellt die Firma Sportwagen und Luxusautos her.\\
Im Finanzjahr 2018 erzielte der Konzern einen Umsatz von 25,784 Mrd. Euro.\\ %\cite{hier quelle wikipedia einfügen}
Die aktuelle Produktpalette der AG umfasst für Serienfahrzeuge:\\

\begin{itemize}
	\item{Macan (2013)}
	\item{Cayenne (2002)}
	\item{911 (1963)}
	\item{Boxter (1996)}
	\item{Panamera (2009)}
	\item{Taycan (2019)}
 \end{itemize}

\section*{Weissach}

Seit 1971 werden diese Fahrzeuge am Standort Weissach entwickelt. Das sogenannte "Entwicklungszentrum Weissach", kurz EZW, auf dem die Porsche AG ein Fahrzeug entwickeln, testen und justieren kann, bevor es in Serienproduktion in Zuffenhausen oder einem anderen Standort geht. \\
%Hier Bild vom EZW einfügen
Alleine an diesem Standort beschäftigt Porsche um die 6500 Mitarbeiter und Mitarbeiterinnen. Zuletzt wurde hier auch der Porsche 918 entickelt: Ein Supersportwagen mit hybridem Antrieb und einem Listenpreis von mehr als 800 000 Euro.\\
%hier halb-seitiges Bild vom 918
Das in Weissach entwickelt wird, sieht man schon an der Anzahl an Patenten die hier jährlich angemeldet werden. Knapp 400 pro Jahr sind es aktuell, mit mehr als 7000 bereits zugelassenen Patenten.\\
Die Signifikanz des Standortes Weissach wird immer wieder durch spezielle Produktvarianten oder Ausstattungen hervorgehoben. Ein Beispiel hierfür ist das Weissach-Paket, welches man für den 911 bestellen kann. Teil hiervon sind typischerweise die Farben Grün und Weiß, zusammen mit besonderen Aero-Elementen.

%\section*{Porsche als Ausbildungsbetrieb}


